\documentclass{article}
\usepackage{graphicx} 
\usepackage{tabularx}
\usepackage[dutch]{babel}
\begin{document}
\sffamily
\begin{titlepage}
  \centering
    \vfill
    {\bfseries\Huge
      PVA Automated Systems, \\
      Project codename Beezkneez
        \vskip2cm
      }
      {\bfseries\Large
        Mark Steijger\\
      }
      {
        \bfseries\normalsize
        0938713\\
        \vskip1cm
        \today\\
    }    
    \vfill
    \includegraphics[width=4cm]{logohr.png} % also works with logo.pdf
    \vfill
    \vfill
\end{titlepage}
\newpage
\tableofcontents

\newpage

\section{Projectbeschrijving en teamsamenstelling}
Beschrijf in het algemeen hoe het project eruit zal zien en wie de projectleden zijn.  

\subsection{Architectuur}
Leg eerst de architectuur van je project vast: definieer welke type architectuur jullie gekozen hebben, 
maak een diagram en leg de belangrijkste keuzes vast. Denk aan centraal vs decentraal of gemengd (en leg dan goed vast wat waar hoort). 
Maak een tekening van de logische blokken van de architectuur. Denk aan communicatie (tussen de robots, 
tussen de robot en het centrale systeem), robot tracking systeem, sensoriek, collision detection en avoidance, etc… 

\subsection{Swarm opdracht}
Definieer hier welke opdracht de swarm gaat uitvoeren. 
Denk aan minimumeisen voor de opdracht en eventueel uitbreiding (must haves en should haves) 

\subsection{Scope (optioneel)}
Tijdens het project is het mogelijk dat de scope wijzigt (denk maar aan ‘scope creep’) 
wat dus ook invloed heeft op het eindresultaat (budget, deadline, …). Houd deze bij in een log.  



\begin{table}[h]
  \begin{tabularx}{\textwidth}{| >{\raggedright\arraybackslash}X | >{\raggedright\arraybackslash}X | >{\centering\arraybackslash}X |}
  \hline
  Activiteiten binnen scope &  Activiteiten buiten scope & Changelog  \\ \hline
   &  &  \\ \hline
  \end{tabularx}
\end{table}



\section{Taken en verantwoordelijkheden teamleden}
Definieer voor alle teamleden wat hun taken en verantwoordelijkheden zijn binnen het project. 
Dit is belangrijk voor de definitie van het individueel deel.


\begin{table}[h]
  \begin{tabularx}{\textwidth}{| >{\raggedright\arraybackslash}X | >{\raggedright\arraybackslash}X | >{\centering\arraybackslash}X |}
  \hline
  Naam &  Verantwoordelijkheden & Omschijving  \\ \hline
   &  &  \\ \hline
  \end{tabularx}
\end{table}


\section{Fases}
Vervolgens splits je je project op in behapbare onderdelen en fases. De meeste projecten bestaan uit 5 fases: 
opstart, planning, uitvoering, opvolging en controle, beëindiging. Elke fase bevat specifieke milestones en taken, 
die je in één van de volgende stappen bepaalt. 


\begin{table}[h]
  \begin{tabularx}{\textwidth}{| >{\raggedright\arraybackslash}l |  >{\centering\arraybackslash}X |}
  \hline
  Fase &  Omschrijving   \\ \hline
   &  \\ \hline
  \end{tabularx}
\end{table}


\section{Deliverables en milestones}
Maak een lijst van alle belangrijke items die je moet bezorgen. Voeg deliverables en deadlines toe. 
De volgende stap is om dit op te splitsen in milestones en taken. 


\begin{table}[h]
  \begin{tabularx}{\textwidth}{| >{\raggedright\arraybackslash}X | >{\centering\arraybackslash}X | >{\centering\arraybackslash}X | >{\centering\arraybackslash}X |}
  \hline
  Deliverables &  Omschrijving & Opleverdatum  & Afhankelijkheden\\ \hline
   &  &  & \\ \hline
  \end{tabularx}
\end{table}


\section{Taken}
Maak een lijst van de belangrijkste taken die nodig zijn om een project af te leveren. 
Indien mogelijk, koppel je die taken aan de relevante milestones en duid je de opeenvolging en afhankelijkheid aan: 
moet je eerst een andere taak uitvoeren voordat je met een nieuwe kan starten? 
Vaak zijn er meerdere taken nodig voor verschillende milestones.  

Stel je voor dat een taak vertraging oploopt. Afhankelijkheden helpen je om de impact van die vertraging 
op de uiteindelijke deadline te beoordelen. 


\begin{table}[h]
  \begin{tabularx}{\textwidth}{| >{\raggedright\arraybackslash}X | >{\centering\arraybackslash}X | >{\centering\arraybackslash}X | >{\centering\arraybackslash}X |}
  \hline
  Taak &  Deadline & Milestone  & Wie\\ \hline
   &  &  & \\ \hline
  \end{tabularx}
\end{table}


\section{Kwaliteitszorg}
Omschrijf hoe je de kwaliteit van de deliverables zal beoordelen. Om die goed te meten en beoordelen, 
zorg je opnieuw voor een duidelijk proces met vastomlijnde taken en verantwoordelijkheden. 
Enkele voorbeelden van deze processen zijn: ‘peer reviews’, toetsing van het ontwerp en het testen van het product.  

\section{Hardwarelijst (bestellijst)}
Maak hier de lijst van hardware en andere middelen die je nodig hebt voor het project.  

\begin{table}[h]
  \begin{tabularx}{\textwidth}{| >{\raggedright\arraybackslash}X | >{\centering\arraybackslash}X | >{\centering\arraybackslash}X |}
  \hline
  Naam &  Beschrijving & Indicatie kosten  \\ \hline
   &  &  \\ \hline
  \end{tabularx}
\end{table}


\newpage
\bibliography{references}
\bibliographystyle{plain}
\end{document}


