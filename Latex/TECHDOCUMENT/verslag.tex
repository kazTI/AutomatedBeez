\documentclass{article}
\usepackage{graphicx} 
\usepackage{tabularx}
\usepackage{lscape}
\usepackage{colortbl}
\usepackage[dutch]{babel}
\begin{document}
\sffamily
\begin{titlepage}
  \centering
    \vfill
    {\bfseries\Huge
      PVA Automated Systems, \\
      Project codename Beezkneez
        \vskip2cm
      }
      {\bfseries\Large
        Mark Steijger\\
      }
      {
        \bfseries\normalsize
        0938713\\
            \vskip1cm
    }
          {\bfseries\Large
        Kazimir Piek\\
      }
      {
        \bfseries\normalsize
        0953725\\
            \vskip1cm
    }      {\bfseries\Large
        Robert Karajev\\
      }
      {
        \bfseries\normalsize
        0851997\\
            \vskip1cm
    }      {\bfseries\Large
        Michael Francis\\
      }
      {
        \bfseries\normalsize
        0963038\\
            \vskip1cm
    }
            \vskip1cm
        \today\\
    \vfill
    \includegraphics[width=4cm]{../IMAGES/logohr.png} % also works with logo.pdf
    \vfill
    \vfill
\end{titlepage}
\newpage
\tableofcontents

\newpage
\section{Inleiding}
Dit document dient als een technisch naslagwerk voor AutomatedBeezzzz project, hierin worden de verschillende aspecten
die nodig zijn geweest voor het uiteindelijk complete product besproken.

\section{System/Subsystem Specification}


\subsection{Identification}
Het volledige systeem bestaat uit de volgende subsystemen.
\begin{enumerate}
    \item Camera
    \item Server
    \item Simulatie
\end{enumerate}

\subsection{System overview}
Het doel van het systeem is het nabootsen van het gedrag dat bijen uitvoeren bij het verzamelen van eten.
Hierbij wordt voor de bijen gebruik gemaakt van zowel hardware als gesimuleerde bijen.
Deze bijen zullen vanuit hun start locatie, de korf, eten zoeken
en hierna vervolgens andere bijen op de hoogte brengen van de locatie van deze voedselbron.

Het project zal uitgevoerd worden door de AutomatedBeezzzz projectgroep

\subsection{Document overview}
Dit document is geschreven om het gehele systeem van de drone simulatie die de bijen nabootst te omschrijven. Hierbij komt kijken de architectuur, eisen, designs en een testplan waar het systeem aan moet voldoen.

\subsection{Referenced documents}
Versies van dit document:
\begin{itemize}
    \item Versie 1: 6-19-21
\end{itemize}

\section{System/Subsystem Design Description}
Aan de camera zit een aantal eisen. Het meest voor de hand liggende eis is dat de resolutie hoog genoeg moet zijn
om bepaalde objecten te kunnen herkennen. Dit is niet de enige eis, zo moet de camera ook kleur kunnen herkennen 
en modulair zijn. Deze camera zal aangesloten moeten worden aan een microcontroller 
zodat de fotos die de camera maakt bewerkt kan worden. Dit betekent dat de microcontroller krachtig genoeg moet zijn 
om fotos op te vragen, verwerken en berekeningen erop uitvoeren. Ook moet hij vervolgens de verwerkte data doorsturen 
naar een centrale server. Dit betekent dat er een manier moet zijn op de microcontroller om deze data te verzenden.
Dit zijn de hardware eisen aan de camera kant, als dit allemaal voldaan wordt is het mogelijk om de camera te 
gebruiken voor de simulatie van bijen.


Uit onderzoek is gebleken dat de Picam V2 voldoet aan de eisen die is gesteld bij de SSS. 
Om deze reden hebben wij voor deze camera gekozen. Elke moderne webcam zou werken maar communicatie met een 
microcontroller wordt onnodig ingewikkeld. Ook zijn deze webcams van een hogere prijsklasse.
Als er gekozen wordt voor een Picam voor de camera, is het uiteraard gebruikelijk om dan een raspberry pi 
te gebruiken als microcontroller. Dit is de enige microcontroller die plug and play werkt met een Picam. 
Vervolgens wordt er gekeken naar de eisen van de Pi, en er is gebleken dat deze eraan voldoet. 
De Pi 4 is krachtig genoeg om afbeeldingen te verwerken en heeft een ingebouwde bluetooth module 
om deze data weer op te sturen naar de server. Dit is dus waarom er gekozen is voor een Raspberry Pi 4.

\subsection{System wide design decisions}
In dit hoofdstuk worden, zoals de tital al verraad, de design decisions behandeld. Hieronder vallen bijvoorbeeld 
input/outputs van het systeem, gedrag keuzes en andere onderdelen die voor het gehele systeem gelden


\subsubsection*{Camera}

\subsubsection*{Server}
asd
\newline

\subsubsection*{Simulatie}
    \begin{center}
    \hspace*{-3.5cm}\includegraphics[width=1.6\textwidth]{../IMAGES/Simulation Interaction.jpg}
    \end{center}
    Het diagram hierboven weergeeft de communicatie tussen de simulatie en de server. In de simulatie
    wordt het gedrag van bijen met behulp van 3 ontworpen drones gesimuleerd. Om dat mogelijk te maken
    zijn voor de input en de output de volgende ontwerp keuzes gemaakt: \\
    
    \begin{description}\setlength{\itemindent}{0.1cm}
        \item[INPUT:] \hfill
        \begin{enumerate}
            \item De input voor de simulatie is altijd een opdracht naar een controller. Deze opdracht is een
            actie die een drone moet uitvoeren waarbij een positie meegegeven wordt  naarr de volgende verplaatsingsstap.
            Dit wordt mogelijk gemaakt door te communiceren tegen een opgebouwde interface. Deze 
            interface communiceerd dan vervolgens tegen een controller die een gesimuleerde drone bestuurd.
        \end{enumerate}
        \newpage
        \item[OUTPUT:] \hfill
        \begin{enumerate}
            \item De output bij deze simulatie is verplaatsing van de gesimuleerde drone naar een geplannde positie.
            \item Daarnaast is er ook een andere continue output van de simulatie. Dat is een bericht die afkomstig 
            is van een controller. Een controller stuurt om de bepaalde tijd een bericht naar de server met 
            de informatie over de positie van de bestuurde drone.
        \end{enumerate} 
        
        
    \end{description}

    \newpage
\subsection{System architectural design}


\subsubsection*{System components}
Diagram met hardware en software componenten van het gehelen systeem en de relatie tussen deze onderdelen
% \section{Software Design Description}
In dit hoofdstuk worden de diagrammen behandeld die over het systeem of zijn subsystemen gaan
\section{Software Requirement Spesification}

\subsection{requirements}

\subsubsection*{Required states and modes}

\subsubsection*{System internal interface requirements}

\subsubsection*{System internal data requirements}

\subsubsection*{Design and construction constraints}

% \section{Interface Design Description}
Dit hoofstuk behandeld in detail alle interfaces en communicatie de het systeem
en daarmee subsystemen met elkaar hebben.

\subsection{Interface design}

\subsection{Interface identification and diagrams}
\section{Software Test Plan}

\subsection{Software}

\subsection{Hardware}
\section{System Test Report}
In dit hoofstuk worden de testresultaten behandeld, de testplannen hiervoor staan in het vorige hoofdstuk genaamd System test plan.

\subsection{MQTT communicatie}
Voor het testen van de communicatie MQTT, zijn aparte test files geschreven. De messages worden gevormd in de ene test file genaamd test\_publish.py. Deze kunnen vervolgens succesvol worden uitgelezen in test\_recieve.py.

\subsection{Camera object tracking}
De camera kan in een live beeld de drone onderscheiden ende locatie van deze drone aangeven. Vervolgens worden deze pixel coordinaten omgevormd naar grid coordinaten die doorgestuurd worden. In de afbeelding is te zien hoe de drone wordt onderscheiden van zijn omgeving en coordinaten aan hem worden toegepast.

\subsection{Drone Foodsearch}
De crazyflie heeft soms problemen met opstijgen door een lage batterij, maar zolang de batterij minstens 25 procent batterij heeft is er niks aan de hand en vliegt de drone, via kleine, snelle stappen, naar de voedselbron volgens de route die pathfinding geplanned heeft. Zodra de drone geland is op de voedselbron stijgt hij weer op om terug te keren naar de korf, als geariveerd begint hij met dansen door 360 graden links en dan 360 graden rechts te draaien. Na het dansen land hij in de korf, mogelijk is hij een beetje verplaatst door het dansen, maar niet zoveel dat hij buiten de korf zou komen.
Deze test is ook te zien in filmpje dat hoort bij de opleverset.

\subsection{Drone Foodget}
Het gedrag van de drone is voor het eerste vlieggedeelte hetzelfde als bij de Foodsearch. Echter zodra hij bij de korf arriveert land hij en zolang de voedselbron nog niet leeg is stijgt hij weer op om het te herhalen.
Dit gedeelte van de test werkt, maar de collision avoidance werkt op het moment van het inleveren van dit rapport helaas nog niet, dit zal in de toekomst nog gedaan worden.

\subsection{Crazyflie weergeven in simulatie}
De informatie voor de locatie van de Crazyflie wordt doorgegeven door middel van MQTT messages. Met de binnengekomen berichten wordt het verschil in thuis en verplaasting uitgerekend zodat de beweging in de simulatie vloeiend is. Deze test is te zien in de bijgeleverde video in de opleveset.
\section{Risico Analyse}
\begin{landscape}
    \begin{table}[ht]
        \centering
        \scalebox{1}{
            \begin{tabular}{|l|>{\centering\arraybackslash}m{6cm}|c|c|>{\centering\arraybackslash}m{6cm}|c|c|}
            \hline
            Nr. & Risico's & Kans & Gevolg & Maatregelen & Kans & Gevolg \\
            \hline
            \hline
            1 & Niet genoeg budget voor het realiseren van een Swarm. & 8 & \cellcolor{yellow}middel & Alternatieve oplossing vinden voor het realiseren van een swarm & 1 & \cellcolor{yellow}middel \\
            \hline
            2 & Breken van propellers tijdens vliegen van drone. & 5 & \cellcolor{yellow}middel & Nauwkeurig te werk gaan en de drone gecontroleerd besturen & 3 & \cellcolor{yellow}middel \\
            \hline
            3 & Webots omgeving is niet compatabel met programmeer taal Python. & 7 & \cellcolor{red}hoog & Uitzoeken of Webots mogelijk te gebruiken is met Python. & 5 & \cellcolor{yellow}middel \\
            \hline
            4 & Te weinig kennis/vaardigheden voor het ontwikkelen van kleur detectie algorithme. & 7 & \cellcolor{yellow} middel & Uitzoeken of het mogelijk is om gebruik te maken van kant en klare libraries. & 1 & \cellcolor{green}laag \\
            \hline
            5 & Niet op elkaar afgestemde losse gebouwde componenten. & 9 & \cellcolor{red}hoog & Vooraf afspreken hoe het gehele systeem architectuur uit komt te zien. & 4 & \cellcolor{yellow}middel \\
            \hline
            6 & Besturing van de drone is niet genoeg om het volgens grid coordinaten systeem te laten vliegen. & 6 & \cellcolor{yellow}middel & Uitzoeken of crazyflies iets hebben waarbij positie bepaling of stabilizatie mee gedaan wordt. & 2 & \cellcolor{yellow}middel \\
            \hline
            7 & Niet bruikbare MQTT communicatie tussen simulatie en de server. & 6 & \cellcolor{green}laag & Uitproberen welke tijdsintervallen het beste prestatie leveren. & 1 & \cellcolor{green}laag \\
            \hline
            8 & Inadequate swarm gedrag realiseren. & 5 & \cellcolor{red}hoog & De tijd nemen om swarm gedrag uit te denken. De scenario die de swarm uit moet voeren in schema zetten. & 1 & \cellcolor{yellow}middel \\
            \hline
            9 & Te lange wachttijd bij levering van hardware componenten. & 8 & \cellcolor{red}hoog & Langs andere groepen gaan om te kijken of bij hun wat te lenen valt. & 4 & \cellcolor{yellow}middel \\
            \hline
            10 & Niet op tijd inleveren van opleverset & 5 & \cellcolor{red}hoog & Ruim op tijd een dag afspreken om er voor te zorgen dat alle documentatie voor de swarm project af is. & 4 & \cellcolor{red}hoog \\
            \hline
            \end{tabular}}
            \caption{Risico analyse voor project AutomatedBeezzzz voor Tinlab Automated Systems}

    \end{table}
\end{landscape}

\section{Conclusie}
De uiteindelijke doestelling, die als eerste werd opgesteld is behaald. Al zijn er wel een hoop onderdelen van het project waar de ideeen die in eerste instantie gemaakt waren, veranderd moesten worden om verschillende redenen. Allereerst is het geen decentralised systeem meer omdat dit te ingewikkeld bleek binnen de tijd. Dit had ook verder geen toegevoegde waarde om de swarm taak uit te voeren, anders dan dat er meer informatie verstuurt moest worden. Ten tweede moest het vloeiend naar locaties vliegen ook op een lagere prioriteit gesteld worden. Dit had namelijk ook geen toegevoegde waarde aan het swarm gedeelte van het project. Als laatste was er gekozen voor een elkele voedsel bron en om deze reden is er ook geen potentie aan de voedsel bronnen meer toegepast. 
Het project is volledig succesvol als een swarm, maar zoals hierboven beschreven zit het nog vol met potentie voor uitbreidingen, mocht een individu dit willen doen.

\newpage
\bibliography{references}
\bibliographystyle{plain}
\end{document}


