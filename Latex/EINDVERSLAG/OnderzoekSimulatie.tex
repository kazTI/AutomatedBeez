\subsection{Simulatie onderzoek}
Voor het simuleren van drones is er gekeken naar beschikbare tools waarmee dit gerealiseerd kon worden. Daarbij
zijn verschillende tools naar boven gekomen. Webots, Gazebo, V-REP en zo waren er veel meer te vinden. Daarnaast is
gekeken naar wat voor model hier het beste bij zou passen \cite{Weisberg2013}. Fysiek model wijst het beste te zijn
aangezien verplaatsing van een bij met behulp van een drone gesimuleerd moet worden. De simulatie in deze context 
zal zowel live als constructief zijn. Om dit goed te realiseren is er ook onderzoek gedaan naar wat een 
goede basis zouden kunnen vormen voor het realiseren van zulk simulatie. Daaruit is een paper over swarm simulation system
naar voren gekomen \cite{minar1996swarm}. Deze paper legt uit hoe een simulatie systeem dat in jaren 1996 gebruikt werd opgebouwt was.
Deze paper vormt dan ook een basis denkwijze voor het te realiseren simulatie systeem.