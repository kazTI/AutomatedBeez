\subsection{Image resultaten}
Uit onderzoek is gebleken dat de Picam V2 voldoet aan de eisen die is gesteld in het technische document. Om deze reden hadden wij oorspronkelijk voor deze camera gekozen. Elke moderne webcam zou werken maar communicatie met een microcontroller wordt onnodig ingewikkeld. Dit is op te lossen door een laptop te gebruiken als microcontroller. Als er gekozen wordt voor een Picam voor de camera, is het uiteraard gebruikelijk om dan een raspberry pi te gebruiken als microcontroller. Dit is de enige microcontroller die plug and play werkt met een Picam. Vervolgens wordt er gekeken naar de eisen van de Pi, en er is gebleken dat deze eraan voldoet. De Pi 4 is krachtig genoeg om afbeeldingen te verwerken en heeft een ingebouwde bluetooth module om deze data weer op te sturen naar de server. Toch is er gekozen voor een webcam, gezien deze van dezelfde prijsklasse is. Ook is deze oplossing meer modulair, gezien elke PC dat een webcam heeft deze python code kan runnen, hiervoor is dus niet eens per se Windows nodig. Natuurlijk is een PC of laptop krachtiger dan een Pi, dus reageert het sneller dan een raspberry pi. Dit betekent dat de locatie vaker geupdate wordt en de drone nauwkeuriger zijn pad kan volgen. De beste oplossing die gevonden werd door de groep is een combinatie van een laptop met een webcam.

De beeldherkenning is in python geschreven. Oorspronkelijk was het de bedoeling om kleurherkenning te gebruiken om te onderscheiden tussen de objecten op het veld. Dit werkte goed, maar werd onnodig ingewikkeld gezien er een filter toegepast moest worden voor pixels die wel een bepaalde kleur heeft maar niet van toepassing is. Er is dus gekozen voor beeldherkenning op basis van de contour van de objecten. Het veld wordt toegewezen door de gebruiker en het programma rekent uit waar binnen dat veld het object zich bevindt. Deze data wordt dan opgestuurd naar de server om verder verwerkt te worden.

