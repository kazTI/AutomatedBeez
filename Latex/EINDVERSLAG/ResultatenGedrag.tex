\subsection{Image gedrag}
Uit het onderzoek is gebleken dat het gedrag van de werk bijen onderverdeeld kan worden in 2 belangrijke rollen, het zoeken naar bloemen met nectar en het oogsten van de bloemen. Maar elke bij kan elke rol vullen, dus dat willen wij ook naar voren brengen in onze swarm. Aan het begin kiezen wij een willekeurige bij om als eerste eten te zoeken. Zodra deze bij het eten gearriveerd is, land hij om te bevestigen en dan keert hij terug naar de korf. Zodra hij terug bij de korf is gaat hij eerst dansen om de andere bijen te laten weten dat er in de buurt eten te vinden is. Vervolgens gaan alle beschikbare bijen in 'oogsten' mode en vliegen dus naar het eten toe om een gedeelte van de beschikbare nectar op te halen en terug naar het nest te brengen. Elke eten bron heeft een bepaalde hoeveelheid eten en de bijen zullen blijven oogsten totdat al het eten in de bron vergaard is. Zodra het eten van deze bron op is, word er opnieuw willekeurig een bij gekozen om op zoek te gaan naar eten.
De reden dat we besloten hebben om telkens maar een enkele bij op zoektocht naar eten te sturen is omdat we besloten hebben om af te wijken van het eigenlijke 'zoeken' naar eten en in plaats daarvan gewoon met de camera kijken waar het eten is en de bij direct daarnaar toe sturen met behulp van pathfinding. Dit doen we omdat, als we echt de bijen op zoektocht willen laten gaan, hebben we bepaalde extra hardware nodig dat erg duur word als we dat op meerdere fysieke bijen willen gebruiken. Om deze reden heeft het geen zin om meerdere bijen op zoektocht te laten gaan.