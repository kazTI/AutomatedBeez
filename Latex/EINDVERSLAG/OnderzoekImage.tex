\subsection{Onderzoek Image}
Een cruciaal onderdeel van een bij om eten op te sporen is hun zicht. Om dit te simuleren is er voor een camera gekozen die vanaf een bij, of in het geval van de simulatie een drone, de afstand naar andere objecten berekent. Er is maar een enkele camera nodig voor alle drones, zolang deze in ieder geval een overzicht heeft van elk object binnen de simulatie.
Uiteraard betekent dit dat er onderzoek gedaan moet worden naar de optimale camera voor deze taak. Er zijn niet veel eisen in dit geval, aangezien image processing het mogelijk maakt om de data te verwerken dat nodig is. Met dit in gedachte zijn er twee eisen aan de camera. Ten eerste moet de camera een opname in kleur maken, aangezien er een verschil wordt gemaakt tussen het eten en de bijen door middel van kleur. Daarnaast moet de camera een minimale resolutie hebben van 480x360 px. Dit is de minimale resolutie want minder dan dat zou te weinig data leveren om te verwerken, wat dan leidt tot onnauwkeurige afstanden.
Er zijn een aantal onderzoeken gedaan naar de ideale camera voor een project als deze. Hierin zijn vergelijkingen gemaakt tussen de GoPro, uEye en PiCam v2. Hieruit is gebleken dat de PiCam v2 de beste optie is voor een laag-budget en klein camera die aan onze eisen voldoen.\cite{picamstudies}

