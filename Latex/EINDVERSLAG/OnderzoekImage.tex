\subsection{Onderzoek Image}
Een cruciaal onderdeel van een bij om eten op te sporen is hun zicht. Om dit te simuleren is er voor een camera gekozen die vanaf een bij, of in het geval van de simulatie een drone, de afstand naar andere objecten berekent. Er is maar een enkele camera nodig voor alle drones, zolang deze in ieder geval een overzicht heeft van elk object binnen de simulatie.
Uiteraard betekent dit dat er onderzoek gedaan moet worden naar de optimale camera voor deze taak. Er zijn niet veel eisen in dit geval, aangezien image processing het mogelijk maakt om de data te verwerken dat nodig is. Met dit in gedachte zijn er twee eisen aan de camera. Ten eerste moet de camera een opname in kleur maken, aangezien er een verschil wordt gemaakt tussen het eten en de bijen door middel van kleur. Daarnaast moet de camera een minimale resolutie hebben van 480x360 px. Dit is de minimale resolutie want minder dan dat zou te weinig data leveren om te verwerken, wat dan leidt tot onnauwkeurige afstanden.
Er zijn een aantal onderzoeken gedaan naar de ideale camera voor een project als deze. Hierin zijn vergelijkingen gemaakt tussen de GoPro, uEye en PiCam v2. Hieruit is gebleken dat de PiCam v2 de beste optie is voor een laag-budget en klein camera die aan onze eisen voldoen.\cite{picamstudies}
Deze camera is gekoppeld aan een Raspberry Pi. Dit maakt het mogelijk om hier direct image processing op uit te oefenen. Image processing houdt in dat afbeeldingen worden verwerkt met behulp van de digitale data van een afbeelding. Het wordt gedaan door drie redenen: digitalisatie, verbetering of restauratie en segmentatie voor machine vision.\cite{imageprocessing} In deze toepassing gaat het vooral om de laatste reden, aangezien het nodig is om bepaalde kleuren te herkennen binnen een afbeelding om daar dan een algoritme op uit te voeren. Deze algoritme zal de afstand berekenen tussen een drone en ieder ander object om deze naar de drone te sturen zodat de drone het kan verwerken. Dit gebeurt voor ieder drone, en zo heeft elke "bij" een "zicht" van alle objecten.
Vervolgens is het noodzakelijk om een protocol op te stellen voor het verbinden tussen de PiCam en de drones. Hier is voor bluetooth gekozen aangezien de Raspberry Pi en de drones standaard een bluetooth module hebben, wat communicatie mogelijk maakt. Ook heeft bluetooth een range van binnen de tien meter volgens de officiele Raspberry Pi handleiding, wat al meer dan genoeg is voor dit project. Er is ook geen sprake van interference of noise, gezien alle objecten een line-of-sight hebben met elkaar. Er zijn mogelijkheden besproken zoals UWB, Wi-Fi of Serial, maar al deze opties hebben niet de voordelen die bluetooth heeft.\\
Hiermee is het duidelijk dat de camera, met de bijbehorende image processing software, een kritiek component is van het gehele systeem en om bijen zo realistisch mogelijk te simuleren.
