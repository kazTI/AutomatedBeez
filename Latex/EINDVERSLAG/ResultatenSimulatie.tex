\subsection{Image simulatie}
Voor het realiseren van simulatie is gekozen voor Webots programma. De keuze hiervoor is gemaakt omdat
een simulatie programma  voor deze tinlab geleert moest worden voordat het gebruikt kon worden. 
Dit is gepaard gedaan met vak Connected Systems waarbij Webots basaal behandeld werd. Ook is deze keuze
gemaakt omdat Webots ontwikkelingen in programmeertaal Python ondersteunt aangezien Python als hoofd programmeertaal
binnen dit project gebruikt wordt. Er is voor gekozen om 3 bijen te simuleren. Dit houd in dat in de simulatie programma
3 drones gesimuleerd worden die bijen gedrag zullen vertonen. De paper die in het onderzoek behandeld werdt vormt
hier ook een mooie basis aan. Elke drone in Webots zal een entiteit zijn en elke drone zal dan ook bestuurd worden
door zogenaamde controller die in Webots geprogrammeerd wordt. Verder is er een keuze gemaakt om aan elke controller
dan ook een eigen communicatie link te verbinden. De conrete protocol zal later gekozen worden.